% Options for packages loaded elsewhere
\PassOptionsToPackage{unicode}{hyperref}
\PassOptionsToPackage{hyphens}{url}
\PassOptionsToPackage{dvipsnames,svgnames,x11names}{xcolor}
%
\documentclass[
  letterpaper,
  DIV=11,
  numbers=noendperiod]{scrartcl}

\usepackage{amsmath,amssymb}
\usepackage{iftex}
\ifPDFTeX
  \usepackage[T1]{fontenc}
  \usepackage[utf8]{inputenc}
  \usepackage{textcomp} % provide euro and other symbols
\else % if luatex or xetex
  \usepackage{unicode-math}
  \defaultfontfeatures{Scale=MatchLowercase}
  \defaultfontfeatures[\rmfamily]{Ligatures=TeX,Scale=1}
\fi
\usepackage{lmodern}
\ifPDFTeX\else  
    % xetex/luatex font selection
\fi
% Use upquote if available, for straight quotes in verbatim environments
\IfFileExists{upquote.sty}{\usepackage{upquote}}{}
\IfFileExists{microtype.sty}{% use microtype if available
  \usepackage[]{microtype}
  \UseMicrotypeSet[protrusion]{basicmath} % disable protrusion for tt fonts
}{}
\makeatletter
\@ifundefined{KOMAClassName}{% if non-KOMA class
  \IfFileExists{parskip.sty}{%
    \usepackage{parskip}
  }{% else
    \setlength{\parindent}{0pt}
    \setlength{\parskip}{6pt plus 2pt minus 1pt}}
}{% if KOMA class
  \KOMAoptions{parskip=half}}
\makeatother
\usepackage{xcolor}
\setlength{\emergencystretch}{3em} % prevent overfull lines
\setcounter{secnumdepth}{-\maxdimen} % remove section numbering
% Make \paragraph and \subparagraph free-standing
\makeatletter
\ifx\paragraph\undefined\else
  \let\oldparagraph\paragraph
  \renewcommand{\paragraph}{
    \@ifstar
      \xxxParagraphStar
      \xxxParagraphNoStar
  }
  \newcommand{\xxxParagraphStar}[1]{\oldparagraph*{#1}\mbox{}}
  \newcommand{\xxxParagraphNoStar}[1]{\oldparagraph{#1}\mbox{}}
\fi
\ifx\subparagraph\undefined\else
  \let\oldsubparagraph\subparagraph
  \renewcommand{\subparagraph}{
    \@ifstar
      \xxxSubParagraphStar
      \xxxSubParagraphNoStar
  }
  \newcommand{\xxxSubParagraphStar}[1]{\oldsubparagraph*{#1}\mbox{}}
  \newcommand{\xxxSubParagraphNoStar}[1]{\oldsubparagraph{#1}\mbox{}}
\fi
\makeatother

\usepackage{color}
\usepackage{fancyvrb}
\newcommand{\VerbBar}{|}
\newcommand{\VERB}{\Verb[commandchars=\\\{\}]}
\DefineVerbatimEnvironment{Highlighting}{Verbatim}{commandchars=\\\{\}}
% Add ',fontsize=\small' for more characters per line
\usepackage{framed}
\definecolor{shadecolor}{RGB}{241,243,245}
\newenvironment{Shaded}{\begin{snugshade}}{\end{snugshade}}
\newcommand{\AlertTok}[1]{\textcolor[rgb]{0.68,0.00,0.00}{#1}}
\newcommand{\AnnotationTok}[1]{\textcolor[rgb]{0.37,0.37,0.37}{#1}}
\newcommand{\AttributeTok}[1]{\textcolor[rgb]{0.40,0.45,0.13}{#1}}
\newcommand{\BaseNTok}[1]{\textcolor[rgb]{0.68,0.00,0.00}{#1}}
\newcommand{\BuiltInTok}[1]{\textcolor[rgb]{0.00,0.23,0.31}{#1}}
\newcommand{\CharTok}[1]{\textcolor[rgb]{0.13,0.47,0.30}{#1}}
\newcommand{\CommentTok}[1]{\textcolor[rgb]{0.37,0.37,0.37}{#1}}
\newcommand{\CommentVarTok}[1]{\textcolor[rgb]{0.37,0.37,0.37}{\textit{#1}}}
\newcommand{\ConstantTok}[1]{\textcolor[rgb]{0.56,0.35,0.01}{#1}}
\newcommand{\ControlFlowTok}[1]{\textcolor[rgb]{0.00,0.23,0.31}{\textbf{#1}}}
\newcommand{\DataTypeTok}[1]{\textcolor[rgb]{0.68,0.00,0.00}{#1}}
\newcommand{\DecValTok}[1]{\textcolor[rgb]{0.68,0.00,0.00}{#1}}
\newcommand{\DocumentationTok}[1]{\textcolor[rgb]{0.37,0.37,0.37}{\textit{#1}}}
\newcommand{\ErrorTok}[1]{\textcolor[rgb]{0.68,0.00,0.00}{#1}}
\newcommand{\ExtensionTok}[1]{\textcolor[rgb]{0.00,0.23,0.31}{#1}}
\newcommand{\FloatTok}[1]{\textcolor[rgb]{0.68,0.00,0.00}{#1}}
\newcommand{\FunctionTok}[1]{\textcolor[rgb]{0.28,0.35,0.67}{#1}}
\newcommand{\ImportTok}[1]{\textcolor[rgb]{0.00,0.46,0.62}{#1}}
\newcommand{\InformationTok}[1]{\textcolor[rgb]{0.37,0.37,0.37}{#1}}
\newcommand{\KeywordTok}[1]{\textcolor[rgb]{0.00,0.23,0.31}{\textbf{#1}}}
\newcommand{\NormalTok}[1]{\textcolor[rgb]{0.00,0.23,0.31}{#1}}
\newcommand{\OperatorTok}[1]{\textcolor[rgb]{0.37,0.37,0.37}{#1}}
\newcommand{\OtherTok}[1]{\textcolor[rgb]{0.00,0.23,0.31}{#1}}
\newcommand{\PreprocessorTok}[1]{\textcolor[rgb]{0.68,0.00,0.00}{#1}}
\newcommand{\RegionMarkerTok}[1]{\textcolor[rgb]{0.00,0.23,0.31}{#1}}
\newcommand{\SpecialCharTok}[1]{\textcolor[rgb]{0.37,0.37,0.37}{#1}}
\newcommand{\SpecialStringTok}[1]{\textcolor[rgb]{0.13,0.47,0.30}{#1}}
\newcommand{\StringTok}[1]{\textcolor[rgb]{0.13,0.47,0.30}{#1}}
\newcommand{\VariableTok}[1]{\textcolor[rgb]{0.07,0.07,0.07}{#1}}
\newcommand{\VerbatimStringTok}[1]{\textcolor[rgb]{0.13,0.47,0.30}{#1}}
\newcommand{\WarningTok}[1]{\textcolor[rgb]{0.37,0.37,0.37}{\textit{#1}}}

\providecommand{\tightlist}{%
  \setlength{\itemsep}{0pt}\setlength{\parskip}{0pt}}\usepackage{longtable,booktabs,array}
\usepackage{calc} % for calculating minipage widths
% Correct order of tables after \paragraph or \subparagraph
\usepackage{etoolbox}
\makeatletter
\patchcmd\longtable{\par}{\if@noskipsec\mbox{}\fi\par}{}{}
\makeatother
% Allow footnotes in longtable head/foot
\IfFileExists{footnotehyper.sty}{\usepackage{footnotehyper}}{\usepackage{footnote}}
\makesavenoteenv{longtable}
\usepackage{graphicx}
\makeatletter
\def\maxwidth{\ifdim\Gin@nat@width>\linewidth\linewidth\else\Gin@nat@width\fi}
\def\maxheight{\ifdim\Gin@nat@height>\textheight\textheight\else\Gin@nat@height\fi}
\makeatother
% Scale images if necessary, so that they will not overflow the page
% margins by default, and it is still possible to overwrite the defaults
% using explicit options in \includegraphics[width, height, ...]{}
\setkeys{Gin}{width=\maxwidth,height=\maxheight,keepaspectratio}
% Set default figure placement to htbp
\makeatletter
\def\fps@figure{htbp}
\makeatother

\usepackage{fvextra}
\usepackage{unicode-math}
\DefineVerbatimEnvironment{Highlighting}{Verbatim}{breaklines,commandchars=\\\{\}}
\DefineVerbatimEnvironment{OutputCode}{Verbatim}{breaklines,commandchars=\\\{\}}
\KOMAoption{captions}{tableheading}
\makeatletter
\@ifpackageloaded{caption}{}{\usepackage{caption}}
\AtBeginDocument{%
\ifdefined\contentsname
  \renewcommand*\contentsname{Table of contents}
\else
  \newcommand\contentsname{Table of contents}
\fi
\ifdefined\listfigurename
  \renewcommand*\listfigurename{List of Figures}
\else
  \newcommand\listfigurename{List of Figures}
\fi
\ifdefined\listtablename
  \renewcommand*\listtablename{List of Tables}
\else
  \newcommand\listtablename{List of Tables}
\fi
\ifdefined\figurename
  \renewcommand*\figurename{Figure}
\else
  \newcommand\figurename{Figure}
\fi
\ifdefined\tablename
  \renewcommand*\tablename{Table}
\else
  \newcommand\tablename{Table}
\fi
}
\@ifpackageloaded{float}{}{\usepackage{float}}
\floatstyle{ruled}
\@ifundefined{c@chapter}{\newfloat{codelisting}{h}{lop}}{\newfloat{codelisting}{h}{lop}[chapter]}
\floatname{codelisting}{Listing}
\newcommand*\listoflistings{\listof{codelisting}{List of Listings}}
\makeatother
\makeatletter
\makeatother
\makeatletter
\@ifpackageloaded{caption}{}{\usepackage{caption}}
\@ifpackageloaded{subcaption}{}{\usepackage{subcaption}}
\makeatother

\ifLuaTeX
  \usepackage{selnolig}  % disable illegal ligatures
\fi
\usepackage{bookmark}

\IfFileExists{xurl.sty}{\usepackage{xurl}}{} % add URL line breaks if available
\urlstyle{same} % disable monospaced font for URLs
\hypersetup{
  pdftitle={HW 2},
  pdfauthor={Bryan Mui - UID 506021334 - 28 April 2025},
  colorlinks=true,
  linkcolor={blue},
  filecolor={Maroon},
  citecolor={Blue},
  urlcolor={Blue},
  pdfcreator={LaTeX via pandoc}}


\title{HW 2}
\author{Bryan Mui - UID 506021334 - 28 April 2025}
\date{}

\begin{document}
\maketitle


Loaded packages: ggplot2, tidyverse (include = false for this chunk)

Reading the dataset:

\begin{Shaded}
\begin{Highlighting}[]
\NormalTok{data }\OtherTok{\textless{}{-}} \FunctionTok{read\_csv}\NormalTok{(}\StringTok{"dataset{-}logistic{-}regression.csv"}\NormalTok{)}
\end{Highlighting}
\end{Shaded}

\begin{verbatim}
Rows: 10000 Columns: 101
-- Column specification --------------------------------------------------------
Delimiter: ","
dbl (101): y, X1, X2, X3, X4, X5, X6, X7, X8, X9, X10, X11, X12, X13, X14, X...

i Use `spec()` to retrieve the full column specification for this data.
i Specify the column types or set `show_col_types = FALSE` to quiet this message.
\end{verbatim}

\begin{Shaded}
\begin{Highlighting}[]
\FunctionTok{head}\NormalTok{(data, }\AttributeTok{n =} \DecValTok{25}\NormalTok{)}
\end{Highlighting}
\end{Shaded}

\begin{verbatim}
# A tibble: 25 x 101
       y      X1      X2     X3     X4     X5     X6      X7     X8      X9
   <dbl>   <dbl>   <dbl>  <dbl>  <dbl>  <dbl>  <dbl>   <dbl>  <dbl>   <dbl>
 1     1 -0.0895  0.450   1.71   0.657 -0.392  1.24   0.895   1.13  -0.0117
 2     1 -0.0943  0.281  -0.147 -0.701  0.400 -0.210  0.677  -0.440  0.458 
 3     0 -0.431  -0.445  -0.777 -0.832 -2.26  -1.62  -1.98   -1.67  -1.15  
 4     0  0.644   0.0817 -0.448  0.852 -1.02   0.671  0.299   0.145 -0.205 
 5     1 -0.919  -0.0241  0.807 -0.612 -0.498  0.350  1.12    0.242 -0.947 
 6     0 -1.89   -1.11   -0.210  0.161 -1.34  -2.04  -0.0135 -1.39  -1.31  
 7     0 -1.34   -0.804   0.322 -0.110  0.624 -0.329 -0.432  -0.191  0.171 
 8     1  0.329   0.468   0.719  0.588  1.71   1.39   0.603   0.650  0.161 
 9     0  0.332   1.42   -0.431  1.02   0.484  0.348  0.474   1.26  -0.479 
10     0 -0.311   0.0193  0.168 -0.346  0.626 -0.704 -0.290   0.680 -0.0453
# i 15 more rows
# i 91 more variables: X10 <dbl>, X11 <dbl>, X12 <dbl>, X13 <dbl>, X14 <dbl>,
#   X15 <dbl>, X16 <dbl>, X17 <dbl>, X18 <dbl>, X19 <dbl>, X20 <dbl>,
#   X21 <dbl>, X22 <dbl>, X23 <dbl>, X24 <dbl>, X25 <dbl>, X26 <dbl>,
#   X27 <dbl>, X28 <dbl>, X29 <dbl>, X30 <dbl>, X31 <dbl>, X32 <dbl>,
#   X33 <dbl>, X34 <dbl>, X35 <dbl>, X36 <dbl>, X37 <dbl>, X38 <dbl>,
#   X39 <dbl>, X40 <dbl>, X41 <dbl>, X42 <dbl>, X43 <dbl>, X44 <dbl>, ...
\end{verbatim}

Our data set has 10000 observations, 1 binary outcome variable y, and
100 predictor variables X1-X100

Separating into X matrix and y vector:

\begin{Shaded}
\begin{Highlighting}[]
\NormalTok{X }\OtherTok{\textless{}{-}}\NormalTok{ data }\SpecialCharTok{\%\textgreater{}\%}
  \FunctionTok{select}\NormalTok{(}\SpecialCharTok{{-}}\NormalTok{y)}
\NormalTok{y }\OtherTok{\textless{}{-}}\NormalTok{ data }\SpecialCharTok{\%\textgreater{}\%}
  \FunctionTok{select}\NormalTok{(y)}
\end{Highlighting}
\end{Shaded}

\section{Problem 1}\label{problem-1}

\subsection{\texorpdfstring{Part
(\(\symbf{\alpha}\))}{Part (\textbackslash symbf\{\textbackslash alpha\})}}\label{part-symbfalpha}

The optimization problem is to minimize the log-likelihood function.
From there we will get the objective function and gradient function

From the slides in class we have:

\[
\min_{\beta} (-\ell(\beta)) = \frac{1}{m} \sum_{i=1}^{m} f_i(\beta)
\]

and the equation for \(f_i(\beta)\):

\[
f_i(\beta) = -y_i(x^\intercal\beta) + log(1 + exp(x_i^\intercal \beta))
\] For the objective function, we get:

\[
\boxed{f(\beta) = \frac{1}{m} \sum_{i=1}^{m} [-y_i(x^\intercal\beta) + log(1 + exp(x_i^\intercal \beta))]}
\]

We also have the gradient function:

\[
\nabla f(x) = \frac{1}{m} \sum_{i=1}^m \nabla f_i(x)
\]

and

\[
\nabla_\beta f_i(\beta) = [\sigma(x_i^\intercal \beta) - y_i] \cdot x_i
\] where \(\sigma(z) = \frac{1}{1+exp(-z)}\) as the logistic sigmoid
function, therefore:

\[
\begin{aligned}
\nabla f(x) &= \frac{1}{m} \sum_{i=1}^m \nabla f_i(x), \; \nabla_\beta f_i(\beta) = [\sigma(x_i^\intercal \beta) - y_i] \cdot x_i \\
\nabla f(\beta) &= \boxed{{\frac{1}{m} \sum_{i=1}^m [\sigma(x_i^\intercal \beta) - y_i] \cdot x_i}}
\end{aligned}
\]

Therefore our gradient descent update step is:

\[
\boxed {\beta_{k+1} = \beta_k - \eta \nabla f(\beta_k)}
\]

\textbf{Implement the following algorithms to obtain estimates of the
regression coefficients \(\symbf{β}\):}

\subsubsection{(1) Gradient descent with backtracking line
search}\label{gradient-descent-with-backtracking-line-search}

Algorithm; Backtracking Line Search:

Params:

\begin{itemize}
\tightlist
\item
  Set \(η^0 > 0\)(usually a large value \textasciitilde1),
\item
  Set \(η_1 = η^0\)
\item
  Set \(ϵ ∈ (0,1), τ ∈ (0,1)\), where \(ϵ\) and \(τ\) are used to modify
  step size
\end{itemize}

Repeat:

\begin{itemize}
\tightlist
\item
  At iteration k, set \(η_k <- η_{k-1}\)

  \begin{enumerate}
  \def\labelenumi{\arabic{enumi}.}
  \item
    Check whether the Armijo Condition holds: \[
    h(η_k) ≤ h(0) + ϵη_kh'(0)
    \]\\
    where \(h(η_k) = f(x_k) − η_k ∇f(x_k)\),\\
    and \(h(0) = f(x_k)\),\\
    and \(h'(0) = -||\nabla (x_k)||^2\)
  \item
  \end{enumerate}

  \begin{itemize}
  \tightlist
  \item
    If yes(condition holds), terminate and keep \(η_k\)
  \item
    If no, set \(η_k = τη_k\) and go to Step 1
  \end{itemize}
\end{itemize}

Stopping criteria: Stop if \(||x_k - x_{k+1}|| ≤ tol\) (change in
parameters is small)

\begin{Shaded}
\begin{Highlighting}[]
\CommentTok{\# logistic gradient descent w/ bls}
\NormalTok{log\_bls }\OtherTok{\textless{}{-}} \ControlFlowTok{function}\NormalTok{(X, y, }\AttributeTok{eta =} \ConstantTok{NULL}\NormalTok{, }\AttributeTok{tol =} \FloatTok{1e{-}6}\NormalTok{, }\AttributeTok{max\_iter =} \DecValTok{10000}\NormalTok{, }\AttributeTok{xi =} \FloatTok{0.5}\NormalTok{, }\AttributeTok{epsilon =} \FloatTok{0.5}\NormalTok{, }\AttributeTok{tau =} \FloatTok{0.5}\NormalTok{) \{}
  \CommentTok{\# Initialize}
\NormalTok{  n }\OtherTok{\textless{}{-}} \FunctionTok{nrow}\NormalTok{(X)}
\NormalTok{  p }\OtherTok{\textless{}{-}} \FunctionTok{ncol}\NormalTok{(X)}
\NormalTok{  beta }\OtherTok{\textless{}{-}} \FunctionTok{rep}\NormalTok{(}\DecValTok{0}\NormalTok{, p)}
\NormalTok{  obj\_values }\OtherTok{\textless{}{-}} \FunctionTok{numeric}\NormalTok{(max\_iter)}
\NormalTok{  eta\_values }\OtherTok{\textless{}{-}} \FunctionTok{numeric}\NormalTok{(max\_iter)  }\CommentTok{\# To store eta values used each iteration}
\NormalTok{  beta\_values }\OtherTok{\textless{}{-}} \FunctionTok{list}\NormalTok{() }\CommentTok{\# To store beta values used each iteration}
\NormalTok{  eta\_bt }\OtherTok{\textless{}{-}} \DecValTok{1}  \CommentTok{\# Initial step size for backtracking}
  
  \CommentTok{\# Objective function: negative log{-}likelihood}
\NormalTok{  obj\_function }\OtherTok{\textless{}{-}} \ControlFlowTok{function}\NormalTok{(beta) \{}
    \CommentTok{\#sum((X \%*\% beta {-} y)\^{}2) / (2 * n)}
\NormalTok{  \}}
  
  \CommentTok{\# Gradient function}
\NormalTok{  gradient }\OtherTok{\textless{}{-}} \ControlFlowTok{function}\NormalTok{(beta) \{}
    \CommentTok{\#t(X) \%*\% (X \%*\% beta {-} y) / n}
\NormalTok{  \}}

  \CommentTok{\# Algorithm:}
  \ControlFlowTok{for}\NormalTok{ (iter }\ControlFlowTok{in} \DecValTok{1}\SpecialCharTok{:}\NormalTok{max\_iter) \{}
\NormalTok{    grad }\OtherTok{\textless{}{-}} \FunctionTok{gradient}\NormalTok{(beta)}
\NormalTok{    beta\_values[[iter]] }\OtherTok{\textless{}{-}}\NormalTok{ beta}
    
    \CommentTok{\# backtracking step}
    \ControlFlowTok{if}\NormalTok{ (iter }\SpecialCharTok{==} \DecValTok{1}\NormalTok{) \{}
\NormalTok{      eta\_bt }\OtherTok{\textless{}{-}} \DecValTok{1} \CommentTok{\# Reset only in the first iteration}
\NormalTok{      y\_k }\OtherTok{\textless{}{-}}\NormalTok{ beta}
\NormalTok{    \}}
    \ControlFlowTok{else}\NormalTok{ \{}
\NormalTok{      beta\_prev }\OtherTok{\textless{}{-}}\NormalTok{ beta\_values[[iter }\SpecialCharTok{{-}} \DecValTok{1}\NormalTok{]]}
\NormalTok{      y\_k }\OtherTok{\textless{}{-}}\NormalTok{ beta }\SpecialCharTok{+}\NormalTok{ xi }\SpecialCharTok{*}\NormalTok{ (beta }\SpecialCharTok{{-}}\NormalTok{ beta\_prev)}
\NormalTok{    \}}
\NormalTok{    beta\_new }\OtherTok{\textless{}{-}}\NormalTok{ y\_k }\SpecialCharTok{{-}}\NormalTok{ eta\_bt }\SpecialCharTok{*}\NormalTok{ grad}
    
    \ControlFlowTok{while}\NormalTok{ (}\FunctionTok{obj\_function}\NormalTok{(beta\_new) }\SpecialCharTok{\textgreater{}} \FunctionTok{obj\_function}\NormalTok{(beta) }\SpecialCharTok{{-}}\NormalTok{ epsilon }\SpecialCharTok{*}\NormalTok{ eta\_bt }\SpecialCharTok{*} \FunctionTok{sum}\NormalTok{(grad}\SpecialCharTok{\^{}}\DecValTok{2}\NormalTok{)) \{}
\NormalTok{      eta\_bt }\OtherTok{\textless{}{-}}\NormalTok{ tau }\SpecialCharTok{*}\NormalTok{ eta\_bt}
\NormalTok{      beta\_new }\OtherTok{\textless{}{-}}\NormalTok{ beta }\SpecialCharTok{{-}}\NormalTok{ eta\_bt }\SpecialCharTok{*}\NormalTok{ grad}
\NormalTok{    \}}
\NormalTok{    eta\_used }\OtherTok{\textless{}{-}}\NormalTok{ eta\_bt}
    
\NormalTok{    eta\_values[iter] }\OtherTok{\textless{}{-}}\NormalTok{ eta\_used}
    
\NormalTok{    obj\_values[iter] }\OtherTok{\textless{}{-}} \FunctionTok{obj\_function}\NormalTok{(beta\_new)}
    
    \ControlFlowTok{if}\NormalTok{ (}\FunctionTok{sqrt}\NormalTok{(}\FunctionTok{sum}\NormalTok{((beta\_new }\SpecialCharTok{{-}}\NormalTok{ beta)}\SpecialCharTok{\^{}}\DecValTok{2}\NormalTok{)) }\SpecialCharTok{\textless{}}\NormalTok{ tol) \{}
\NormalTok{      obj\_values }\OtherTok{\textless{}{-}}\NormalTok{ obj\_values[}\DecValTok{1}\SpecialCharTok{:}\NormalTok{iter]}
\NormalTok{      eta\_values }\OtherTok{\textless{}{-}}\NormalTok{ eta\_values[}\DecValTok{1}\SpecialCharTok{:}\NormalTok{iter]}
      \ControlFlowTok{break}
\NormalTok{    \}}
    
\NormalTok{    beta }\OtherTok{\textless{}{-}}\NormalTok{ beta\_new}
\NormalTok{  \}}
  
  \FunctionTok{return}\NormalTok{(}\FunctionTok{list}\NormalTok{(}\AttributeTok{beta =}\NormalTok{ beta, }\AttributeTok{obj\_values =}\NormalTok{ obj\_values, }\AttributeTok{eta\_values =}\NormalTok{ eta\_values, }\AttributeTok{beta\_values =}\NormalTok{ beta\_values))}
\NormalTok{\}}
\end{Highlighting}
\end{Shaded}

\subsubsection{(2) Gradient descent with backtracking line search and
Nesterov
momentum}\label{gradient-descent-with-backtracking-line-search-and-nesterov-momentum}

Nesterov is simply BLS with a special way to select the momentum
\(\epsilon\)

\subsubsection{(3) Gradient descent with AMSGrad-ADAM
momentum}\label{gradient-descent-with-amsgrad-adam-momentum}

(no backtracking line search, since AMSGrad-ADAM adjusts step sizes per
parameter using momentum and adaptive scaling)

\subsubsection{(4) Stochastic gradient descent with a fixed schedule of
decreasing step
sizes}\label{stochastic-gradient-descent-with-a-fixed-schedule-of-decreasing-step-sizes}

\subsubsection{(5) Stochastic gradient descent with AMSGrad-ADAM-W
momentum}\label{stochastic-gradient-descent-with-amsgrad-adam-w-momentum}

\subsection{Part (a)}\label{part-a}

\subsection{Part (b)}\label{part-b}




\end{document}
