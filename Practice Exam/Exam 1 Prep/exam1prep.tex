% Options for packages loaded elsewhere
\PassOptionsToPackage{unicode}{hyperref}
\PassOptionsToPackage{hyphens}{url}
\PassOptionsToPackage{dvipsnames,svgnames,x11names}{xcolor}
%
\documentclass[
  letterpaper,
  DIV=11,
  numbers=noendperiod]{scrartcl}

\usepackage{amsmath,amssymb}
\usepackage{iftex}
\ifPDFTeX
  \usepackage[T1]{fontenc}
  \usepackage[utf8]{inputenc}
  \usepackage{textcomp} % provide euro and other symbols
\else % if luatex or xetex
  \usepackage{unicode-math}
  \defaultfontfeatures{Scale=MatchLowercase}
  \defaultfontfeatures[\rmfamily]{Ligatures=TeX,Scale=1}
\fi
\usepackage{lmodern}
\ifPDFTeX\else  
    % xetex/luatex font selection
\fi
% Use upquote if available, for straight quotes in verbatim environments
\IfFileExists{upquote.sty}{\usepackage{upquote}}{}
\IfFileExists{microtype.sty}{% use microtype if available
  \usepackage[]{microtype}
  \UseMicrotypeSet[protrusion]{basicmath} % disable protrusion for tt fonts
}{}
\makeatletter
\@ifundefined{KOMAClassName}{% if non-KOMA class
  \IfFileExists{parskip.sty}{%
    \usepackage{parskip}
  }{% else
    \setlength{\parindent}{0pt}
    \setlength{\parskip}{6pt plus 2pt minus 1pt}}
}{% if KOMA class
  \KOMAoptions{parskip=half}}
\makeatother
\usepackage{xcolor}
\setlength{\emergencystretch}{3em} % prevent overfull lines
\setcounter{secnumdepth}{-\maxdimen} % remove section numbering
% Make \paragraph and \subparagraph free-standing
\makeatletter
\ifx\paragraph\undefined\else
  \let\oldparagraph\paragraph
  \renewcommand{\paragraph}{
    \@ifstar
      \xxxParagraphStar
      \xxxParagraphNoStar
  }
  \newcommand{\xxxParagraphStar}[1]{\oldparagraph*{#1}\mbox{}}
  \newcommand{\xxxParagraphNoStar}[1]{\oldparagraph{#1}\mbox{}}
\fi
\ifx\subparagraph\undefined\else
  \let\oldsubparagraph\subparagraph
  \renewcommand{\subparagraph}{
    \@ifstar
      \xxxSubParagraphStar
      \xxxSubParagraphNoStar
  }
  \newcommand{\xxxSubParagraphStar}[1]{\oldsubparagraph*{#1}\mbox{}}
  \newcommand{\xxxSubParagraphNoStar}[1]{\oldsubparagraph{#1}\mbox{}}
\fi
\makeatother


\providecommand{\tightlist}{%
  \setlength{\itemsep}{0pt}\setlength{\parskip}{0pt}}\usepackage{longtable,booktabs,array}
\usepackage{calc} % for calculating minipage widths
% Correct order of tables after \paragraph or \subparagraph
\usepackage{etoolbox}
\makeatletter
\patchcmd\longtable{\par}{\if@noskipsec\mbox{}\fi\par}{}{}
\makeatother
% Allow footnotes in longtable head/foot
\IfFileExists{footnotehyper.sty}{\usepackage{footnotehyper}}{\usepackage{footnote}}
\makesavenoteenv{longtable}
\usepackage{graphicx}
\makeatletter
\def\maxwidth{\ifdim\Gin@nat@width>\linewidth\linewidth\else\Gin@nat@width\fi}
\def\maxheight{\ifdim\Gin@nat@height>\textheight\textheight\else\Gin@nat@height\fi}
\makeatother
% Scale images if necessary, so that they will not overflow the page
% margins by default, and it is still possible to overwrite the defaults
% using explicit options in \includegraphics[width, height, ...]{}
\setkeys{Gin}{width=\maxwidth,height=\maxheight,keepaspectratio}
% Set default figure placement to htbp
\makeatletter
\def\fps@figure{htbp}
\makeatother

\usepackage{fvextra}
\usepackage{amssymb}
\usepackage{unicode-math}
\DefineVerbatimEnvironment{Highlighting}{Verbatim}{breaklines,commandchars=\\\{\}}
\DefineVerbatimEnvironment{OutputCode}{Verbatim}{breaklines,commandchars=\\\{\}}
\KOMAoption{captions}{tableheading}
\makeatletter
\@ifpackageloaded{caption}{}{\usepackage{caption}}
\AtBeginDocument{%
\ifdefined\contentsname
  \renewcommand*\contentsname{Table of contents}
\else
  \newcommand\contentsname{Table of contents}
\fi
\ifdefined\listfigurename
  \renewcommand*\listfigurename{List of Figures}
\else
  \newcommand\listfigurename{List of Figures}
\fi
\ifdefined\listtablename
  \renewcommand*\listtablename{List of Tables}
\else
  \newcommand\listtablename{List of Tables}
\fi
\ifdefined\figurename
  \renewcommand*\figurename{Figure}
\else
  \newcommand\figurename{Figure}
\fi
\ifdefined\tablename
  \renewcommand*\tablename{Table}
\else
  \newcommand\tablename{Table}
\fi
}
\@ifpackageloaded{float}{}{\usepackage{float}}
\floatstyle{ruled}
\@ifundefined{c@chapter}{\newfloat{codelisting}{h}{lop}}{\newfloat{codelisting}{h}{lop}[chapter]}
\floatname{codelisting}{Listing}
\newcommand*\listoflistings{\listof{codelisting}{List of Listings}}
\makeatother
\makeatletter
\makeatother
\makeatletter
\@ifpackageloaded{caption}{}{\usepackage{caption}}
\@ifpackageloaded{subcaption}{}{\usepackage{subcaption}}
\makeatother

\ifLuaTeX
  \usepackage{selnolig}  % disable illegal ligatures
\fi
\usepackage{bookmark}

\IfFileExists{xurl.sty}{\usepackage{xurl}}{} % add URL line breaks if available
\urlstyle{same} % disable monospaced font for URLs
\hypersetup{
  pdftitle={Exam 1 Prep(Stats102B)},
  colorlinks=true,
  linkcolor={blue},
  filecolor={Maroon},
  citecolor={Blue},
  urlcolor={Blue},
  pdfcreator={LaTeX via pandoc}}


\title{Exam 1 Prep(Stats102B)}
\author{}
\date{}

\begin{document}
\maketitle


\section{Derivative Rules:}\label{derivative-rules}

\subsection{📘 Basic Derivative Rules}\label{basic-derivative-rules}

\begin{itemize}
\item
  \textbf{Constant Rule}\\
  \[
  \frac{d}{dx}[c] = 0
  \]
\item
  \textbf{Power Rule}\\
  \[
  \frac{d}{dx}[x^n] = nx^{n-1} \quad \text{for any real } n
  \]
\item
  \textbf{Constant Multiple Rule}\\
  \[
  \frac{d}{dx}[c \cdot f(x)] = c \cdot \frac{d}{dx}[f(x)]
  \]
\item
  \textbf{Sum and Difference Rule}\\
  \[
  \frac{d}{dx}[f(x) \pm g(x)] = \frac{d}{dx}[f(x)] \pm \frac{d}{dx}[g(x)]
  \]
\item
  \textbf{Product Rule}\\
  \[
  \frac{d}{dx}[f(x) \cdot g(x)] = f'(x)g(x) + f(x)g'(x)
  \]
\item
  \textbf{Quotient Rule}\\
  \[
  \frac{d}{dx}\left[\frac{f(x)}{g(x)}\right] = \frac{f'(x)g(x) - f(x)g'(x)}{[g(x)]^2}
  \]
\end{itemize}

\subsection{🔁 Chain Rule}\label{chain-rule}

If \(y = f(g(x))\), then:

\[
\frac{dy}{dx} = f'(g(x)) \cdot g'(x)
\]

\subsection{📈 Common Function
Derivatives}\label{common-function-derivatives}

\begin{itemize}
\item
  \textbf{Exponential Functions}\\
  \[
  \frac{d}{dx}[e^x] = e^x
  \] \[
  \frac{d}{dx}[e^{u(x)}] = e^{u(x)} \cdot u'(x)
  \]
\item
  \textbf{Logarithmic Functions}\\
  \[
  \frac{d}{dx}[\ln x] = \frac{1}{x}
  \] \[
  \frac{d}{dx}[\ln(u(x))] = \frac{u'(x)}{u(x)}
  \]
\end{itemize}

\begin{quote}
\emph{Note: Trigonometric derivatives like \(\sin\) and \(\cos\) are not
required.}
\end{quote}

\subsection{🧠 Example: Chain Rule}\label{example-chain-rule}

Let \(f(x) = \ln(3x^2 + 1)\)

Then: \[
f'(x) = \frac{d}{dx}[\ln(3x^2 + 1)] = \frac{6x}{3x^2 + 1}
\]

\section{Practice Exam 1 - FRQs}\label{practice-exam-1---frqs}

\subsection{Problem 1}\label{problem-1}

Let \(f(x) = x - log(x)\),\\
where log(.) donates the natural log base algorithm

\begin{enumerate}
\def\labelenumi{\arabic{enumi}.}
\tightlist
\item
  Show that f(x) has a unique global min. Justify your answer
\end{enumerate}

Find \(f(x), f'(x), f''(x)\), set \(f'(x) = 0\) to get the critical
points, if we can show that \(f'(x) = 0\) has only one critical point,
and that \(f''(x)\) is concave up at that point, then that will be the
unique global min

\[
\begin{aligned}
f(x) &= x - \ln(x) \\
f'(x) &= 1 - \frac{1}{x} \\
0 &= 1 - \frac{1}{x} \\
\frac{1}{x} &= 1 \\
x &= 1 \\
f''(x) &= \frac{1}{x^2} \\
f''(1) &= 1
\end{aligned}
\]

Because \(f(x)\) has one critical point at \(x=1\), and the second
derivative is positive, the function is concave up, meaning that
\(f(x)\) has a unique global minimum at that point.

\begin{enumerate}
\def\labelenumi{\arabic{enumi}.}
\setcounter{enumi}{1}
\item
  Let \(x_0 = 2\) be the initial point used in the gradient descent
  algorithm. What will \(x_1\) be based on the gradient descent
  algorithm, if the step size is set to \(η = 0.5\)?
\item
  Derive the range of values for the step size parameter \(η\), so that
  gradient descent is convergent? Justify your answer.
\item
  Suppose that somebody that does not know how to derive the range of
  eligible step sizes \(η\), decided to use an initial \(η = 2\) for
  initial point \(x_0 = 2\). Explain what calculations the backtracking
  line search algorithm will check to select and appropriate step size
  \(η_0\) to proceed calculating the next update \(x_1\).
\end{enumerate}

\section{Practice Exam 1 - MCQs}\label{practice-exam-1---mcqs}

\subsection{Question 1: Which direction does the gradient descent
algorithm move in each
iter-ation?}\label{question-1-which-direction-does-the-gradient-descent-algorithm-move-in-each-iter-ation}

\begin{itemize}
\tightlist
\item
  \(\Box\) Random direction
\item
  \(\Box\) Direction of the gradient
\item
  \checkbox{1} Opposite to the gradient
\item
  \(\Box\) Along the eigenvectors of the Hessian
\end{itemize}

\subsection{Question 2: If the step size (learning rate) for the
optimization problem of a statistical model is too large, gradient
descent
can:}\label{question-2-if-the-step-size-learning-rate-for-the-optimization-problem-of-a-statistical-model-is-too-large-gradient-descent-can}

\begin{itemize}
\tightlist
\item
  \(\Box\) Converge slowly
\item
  \(\Box\) Not converge
\item
  \(\Box\) Overfit the data
\item
  \(\Box\) Always converge faster
\end{itemize}




\end{document}
