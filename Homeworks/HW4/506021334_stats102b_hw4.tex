% Options for packages loaded elsewhere
\PassOptionsToPackage{unicode}{hyperref}
\PassOptionsToPackage{hyphens}{url}
\PassOptionsToPackage{dvipsnames,svgnames,x11names}{xcolor}
%
\documentclass[
  letterpaper,
  DIV=11,
  numbers=noendperiod]{scrartcl}

\usepackage{amsmath,amssymb}
\usepackage{iftex}
\ifPDFTeX
  \usepackage[T1]{fontenc}
  \usepackage[utf8]{inputenc}
  \usepackage{textcomp} % provide euro and other symbols
\else % if luatex or xetex
  \usepackage{unicode-math}
  \defaultfontfeatures{Scale=MatchLowercase}
  \defaultfontfeatures[\rmfamily]{Ligatures=TeX,Scale=1}
\fi
\usepackage{lmodern}
\ifPDFTeX\else  
    % xetex/luatex font selection
\fi
% Use upquote if available, for straight quotes in verbatim environments
\IfFileExists{upquote.sty}{\usepackage{upquote}}{}
\IfFileExists{microtype.sty}{% use microtype if available
  \usepackage[]{microtype}
  \UseMicrotypeSet[protrusion]{basicmath} % disable protrusion for tt fonts
}{}
\makeatletter
\@ifundefined{KOMAClassName}{% if non-KOMA class
  \IfFileExists{parskip.sty}{%
    \usepackage{parskip}
  }{% else
    \setlength{\parindent}{0pt}
    \setlength{\parskip}{6pt plus 2pt minus 1pt}}
}{% if KOMA class
  \KOMAoptions{parskip=half}}
\makeatother
\usepackage{xcolor}
\usepackage[left=0.3in, right=0.3in, top=0.3in, bottom=0.3in]{geometry}
\setlength{\emergencystretch}{3em} % prevent overfull lines
\setcounter{secnumdepth}{-\maxdimen} % remove section numbering
% Make \paragraph and \subparagraph free-standing
\makeatletter
\ifx\paragraph\undefined\else
  \let\oldparagraph\paragraph
  \renewcommand{\paragraph}{
    \@ifstar
      \xxxParagraphStar
      \xxxParagraphNoStar
  }
  \newcommand{\xxxParagraphStar}[1]{\oldparagraph*{#1}\mbox{}}
  \newcommand{\xxxParagraphNoStar}[1]{\oldparagraph{#1}\mbox{}}
\fi
\ifx\subparagraph\undefined\else
  \let\oldsubparagraph\subparagraph
  \renewcommand{\subparagraph}{
    \@ifstar
      \xxxSubParagraphStar
      \xxxSubParagraphNoStar
  }
  \newcommand{\xxxSubParagraphStar}[1]{\oldsubparagraph*{#1}\mbox{}}
  \newcommand{\xxxSubParagraphNoStar}[1]{\oldsubparagraph{#1}\mbox{}}
\fi
\makeatother

\usepackage{color}
\usepackage{fancyvrb}
\newcommand{\VerbBar}{|}
\newcommand{\VERB}{\Verb[commandchars=\\\{\}]}
\DefineVerbatimEnvironment{Highlighting}{Verbatim}{commandchars=\\\{\}}
% Add ',fontsize=\small' for more characters per line
\usepackage{framed}
\definecolor{shadecolor}{RGB}{241,243,245}
\newenvironment{Shaded}{\begin{snugshade}}{\end{snugshade}}
\newcommand{\AlertTok}[1]{\textcolor[rgb]{0.68,0.00,0.00}{#1}}
\newcommand{\AnnotationTok}[1]{\textcolor[rgb]{0.37,0.37,0.37}{#1}}
\newcommand{\AttributeTok}[1]{\textcolor[rgb]{0.40,0.45,0.13}{#1}}
\newcommand{\BaseNTok}[1]{\textcolor[rgb]{0.68,0.00,0.00}{#1}}
\newcommand{\BuiltInTok}[1]{\textcolor[rgb]{0.00,0.23,0.31}{#1}}
\newcommand{\CharTok}[1]{\textcolor[rgb]{0.13,0.47,0.30}{#1}}
\newcommand{\CommentTok}[1]{\textcolor[rgb]{0.37,0.37,0.37}{#1}}
\newcommand{\CommentVarTok}[1]{\textcolor[rgb]{0.37,0.37,0.37}{\textit{#1}}}
\newcommand{\ConstantTok}[1]{\textcolor[rgb]{0.56,0.35,0.01}{#1}}
\newcommand{\ControlFlowTok}[1]{\textcolor[rgb]{0.00,0.23,0.31}{\textbf{#1}}}
\newcommand{\DataTypeTok}[1]{\textcolor[rgb]{0.68,0.00,0.00}{#1}}
\newcommand{\DecValTok}[1]{\textcolor[rgb]{0.68,0.00,0.00}{#1}}
\newcommand{\DocumentationTok}[1]{\textcolor[rgb]{0.37,0.37,0.37}{\textit{#1}}}
\newcommand{\ErrorTok}[1]{\textcolor[rgb]{0.68,0.00,0.00}{#1}}
\newcommand{\ExtensionTok}[1]{\textcolor[rgb]{0.00,0.23,0.31}{#1}}
\newcommand{\FloatTok}[1]{\textcolor[rgb]{0.68,0.00,0.00}{#1}}
\newcommand{\FunctionTok}[1]{\textcolor[rgb]{0.28,0.35,0.67}{#1}}
\newcommand{\ImportTok}[1]{\textcolor[rgb]{0.00,0.46,0.62}{#1}}
\newcommand{\InformationTok}[1]{\textcolor[rgb]{0.37,0.37,0.37}{#1}}
\newcommand{\KeywordTok}[1]{\textcolor[rgb]{0.00,0.23,0.31}{\textbf{#1}}}
\newcommand{\NormalTok}[1]{\textcolor[rgb]{0.00,0.23,0.31}{#1}}
\newcommand{\OperatorTok}[1]{\textcolor[rgb]{0.37,0.37,0.37}{#1}}
\newcommand{\OtherTok}[1]{\textcolor[rgb]{0.00,0.23,0.31}{#1}}
\newcommand{\PreprocessorTok}[1]{\textcolor[rgb]{0.68,0.00,0.00}{#1}}
\newcommand{\RegionMarkerTok}[1]{\textcolor[rgb]{0.00,0.23,0.31}{#1}}
\newcommand{\SpecialCharTok}[1]{\textcolor[rgb]{0.37,0.37,0.37}{#1}}
\newcommand{\SpecialStringTok}[1]{\textcolor[rgb]{0.13,0.47,0.30}{#1}}
\newcommand{\StringTok}[1]{\textcolor[rgb]{0.13,0.47,0.30}{#1}}
\newcommand{\VariableTok}[1]{\textcolor[rgb]{0.07,0.07,0.07}{#1}}
\newcommand{\VerbatimStringTok}[1]{\textcolor[rgb]{0.13,0.47,0.30}{#1}}
\newcommand{\WarningTok}[1]{\textcolor[rgb]{0.37,0.37,0.37}{\textit{#1}}}

\providecommand{\tightlist}{%
  \setlength{\itemsep}{0pt}\setlength{\parskip}{0pt}}\usepackage{longtable,booktabs,array}
\usepackage{calc} % for calculating minipage widths
% Correct order of tables after \paragraph or \subparagraph
\usepackage{etoolbox}
\makeatletter
\patchcmd\longtable{\par}{\if@noskipsec\mbox{}\fi\par}{}{}
\makeatother
% Allow footnotes in longtable head/foot
\IfFileExists{footnotehyper.sty}{\usepackage{footnotehyper}}{\usepackage{footnote}}
\makesavenoteenv{longtable}
\usepackage{graphicx}
\makeatletter
\newsavebox\pandoc@box
\newcommand*\pandocbounded[1]{% scales image to fit in text height/width
  \sbox\pandoc@box{#1}%
  \Gscale@div\@tempa{\textheight}{\dimexpr\ht\pandoc@box+\dp\pandoc@box\relax}%
  \Gscale@div\@tempb{\linewidth}{\wd\pandoc@box}%
  \ifdim\@tempb\p@<\@tempa\p@\let\@tempa\@tempb\fi% select the smaller of both
  \ifdim\@tempa\p@<\p@\scalebox{\@tempa}{\usebox\pandoc@box}%
  \else\usebox{\pandoc@box}%
  \fi%
}
% Set default figure placement to htbp
\def\fps@figure{htbp}
\makeatother

\usepackage{fvextra}
\DefineVerbatimEnvironment{Highlighting}{Verbatim}{breaklines,commandchars=\\\{\}}
\DefineVerbatimEnvironment{OutputCode}{Verbatim}{breaklines,commandchars=\\\{\}}
\KOMAoption{captions}{tableheading}
\makeatletter
\@ifpackageloaded{caption}{}{\usepackage{caption}}
\AtBeginDocument{%
\ifdefined\contentsname
  \renewcommand*\contentsname{Table of contents}
\else
  \newcommand\contentsname{Table of contents}
\fi
\ifdefined\listfigurename
  \renewcommand*\listfigurename{List of Figures}
\else
  \newcommand\listfigurename{List of Figures}
\fi
\ifdefined\listtablename
  \renewcommand*\listtablename{List of Tables}
\else
  \newcommand\listtablename{List of Tables}
\fi
\ifdefined\figurename
  \renewcommand*\figurename{Figure}
\else
  \newcommand\figurename{Figure}
\fi
\ifdefined\tablename
  \renewcommand*\tablename{Table}
\else
  \newcommand\tablename{Table}
\fi
}
\@ifpackageloaded{float}{}{\usepackage{float}}
\floatstyle{ruled}
\@ifundefined{c@chapter}{\newfloat{codelisting}{h}{lop}}{\newfloat{codelisting}{h}{lop}[chapter]}
\floatname{codelisting}{Listing}
\newcommand*\listoflistings{\listof{codelisting}{List of Listings}}
\makeatother
\makeatletter
\makeatother
\makeatletter
\@ifpackageloaded{caption}{}{\usepackage{caption}}
\@ifpackageloaded{subcaption}{}{\usepackage{subcaption}}
\makeatother

\usepackage{bookmark}

\IfFileExists{xurl.sty}{\usepackage{xurl}}{} % add URL line breaks if available
\urlstyle{same} % disable monospaced font for URLs
\hypersetup{
  pdftitle={Stats 102B HW 4},
  pdfauthor={Bryan Mui - UID 506021334},
  colorlinks=true,
  linkcolor={blue},
  filecolor={Maroon},
  citecolor={Blue},
  urlcolor={Blue},
  pdfcreator={LaTeX via pandoc}}


\title{Stats 102B HW 4}
\author{Bryan Mui - UID 506021334}
\date{}

\begin{document}
\maketitle


Due Wed, June 4, 11:00 pm

\begin{Shaded}
\begin{Highlighting}[]
\FunctionTok{library}\NormalTok{(tidyverse)}
\end{Highlighting}
\end{Shaded}

\begin{verbatim}
-- Attaching core tidyverse packages ------------------------ tidyverse 2.0.0 --
v dplyr     1.1.4     v readr     2.1.5
v forcats   1.0.0     v stringr   1.5.1
v ggplot2   3.5.2     v tibble    3.2.1
v lubridate 1.9.4     v tidyr     1.3.1
v purrr     1.0.4     
-- Conflicts ------------------------------------------ tidyverse_conflicts() --
x dplyr::filter() masks stats::filter()
x dplyr::lag()    masks stats::lag()
i Use the conflicted package (<http://conflicted.r-lib.org/>) to force all conflicts to become errors
\end{verbatim}

\begin{Shaded}
\begin{Highlighting}[]
\NormalTok{train }\OtherTok{\textless{}{-}} \FunctionTok{read\_csv}\NormalTok{(}\StringTok{"train\_data.csv"}\NormalTok{)}
\end{Highlighting}
\end{Shaded}

\begin{verbatim}
Rows: 600 Columns: 601
-- Column specification --------------------------------------------------------
Delimiter: ","
dbl (601): X1, X2, X3, X4, X5, X6, X7, X8, X9, X10, X11, X12, X13, X14, X15,...

i Use `spec()` to retrieve the full column specification for this data.
i Specify the column types or set `show_col_types = FALSE` to quiet this message.
\end{verbatim}

\begin{Shaded}
\begin{Highlighting}[]
\NormalTok{val }\OtherTok{\textless{}{-}} \FunctionTok{read\_csv}\NormalTok{(}\StringTok{"validation\_data.csv"}\NormalTok{)}
\end{Highlighting}
\end{Shaded}

\begin{verbatim}
Rows: 200 Columns: 601
-- Column specification --------------------------------------------------------
Delimiter: ","
dbl (601): X1, X2, X3, X4, X5, X6, X7, X8, X9, X10, X11, X12, X13, X14, X15,...

i Use `spec()` to retrieve the full column specification for this data.
i Specify the column types or set `show_col_types = FALSE` to quiet this message.
\end{verbatim}

\newpage

\subsection{Problem 1}\label{problem-1}

Consider the function

\[
f(x) = \frac{1}{4} x^4 - x^2 + 2x
\]

\subsubsection{\texorpdfstring{Part
\((\alpha)\)}{Part (\textbackslash alpha)}}\label{part-alpha}

Using the pure version of Newton's algorithm report \(x_k\) for
\(k = 20\) (after running the algorithm for 20 iterations) based on the
following 5 initial points:

\begin{enumerate}
\def\labelenumi{\arabic{enumi}.}
\tightlist
\item
  \(x_0 = −1\)
\item
  \(x_0 = 0\)
\item
  \(x_0 = 0.1\)
\item
  \(x_0 = 1\)
\item
  \(x_0 = 2\)
\end{enumerate}

\textbf{Newton's pure algorithm is as follows:}

\begin{enumerate}
\def\labelenumi{\arabic{enumi}.}
\tightlist
\item
  Select \(x_0 \in \mathbb{R}^n\)
\item
  While stopping criterion \textgreater{} tolerance do:

  \begin{enumerate}
  \def\labelenumii{\arabic{enumii}.}
  \tightlist
  \item
    \(x_{k+1} = x_k - [\nabla^2f(x_k)]^{-1} \nabla f(x_k)\)
  \item
    Calculate value of stopping
    criterion(\(|f(x_{k+1}) - f(x_k)| \leq \epsilon\))
  \end{enumerate}
\end{enumerate}

Gradient:
\(\nabla f(x) = \frac{\partial}{\partial x} = f'(x) = x^3 - 2x + 2\)

Hessian:
\(\nabla^2 f(x) = \frac{\partial^2}{\partial x^2} = f''(x) = 3x^2 - 2\)

\begin{Shaded}
\begin{Highlighting}[]
\CommentTok{\# params}
\NormalTok{max\_iter }\OtherTok{\textless{}{-}} \DecValTok{20}
\NormalTok{starting\_points }\OtherTok{\textless{}{-}} \FunctionTok{c}\NormalTok{(}\SpecialCharTok{{-}}\DecValTok{1}\NormalTok{, }\DecValTok{0}\NormalTok{, }\FloatTok{0.1}\NormalTok{, }\DecValTok{1}\NormalTok{, }\DecValTok{2}\NormalTok{)}
\NormalTok{stopping\_tol }\OtherTok{\textless{}{-}} \FloatTok{1e{-}6}


\CommentTok{\# algorithm}
\NormalTok{newton\_pure\_alg }\OtherTok{\textless{}{-}} \ControlFlowTok{function}\NormalTok{(max\_iter, starting\_point, stopping\_tol) \{}
\NormalTok{  beta }\OtherTok{\textless{}{-}}\NormalTok{ starting\_point}
\NormalTok{  iterations\_ran }\OtherTok{\textless{}{-}} \DecValTok{0}
\NormalTok{  betas\_vec }\OtherTok{\textless{}{-}} \FunctionTok{c}\NormalTok{(beta)}
  
\NormalTok{  obj }\OtherTok{\textless{}{-}} \ControlFlowTok{function}\NormalTok{(x) \{}
  \FunctionTok{return}\NormalTok{(}\DecValTok{1}\SpecialCharTok{/}\DecValTok{4} \SpecialCharTok{*}\NormalTok{ x}\SpecialCharTok{\^{}}\DecValTok{4} \SpecialCharTok{{-}}\NormalTok{ x}\SpecialCharTok{\^{}}\DecValTok{2} \SpecialCharTok{+} \DecValTok{2}\SpecialCharTok{*}\NormalTok{x)}
\NormalTok{  \}}
\NormalTok{  grad }\OtherTok{\textless{}{-}} \ControlFlowTok{function}\NormalTok{(x) \{}
\NormalTok{    x}\SpecialCharTok{\^{}}\DecValTok{3} \SpecialCharTok{{-}} \DecValTok{2}\SpecialCharTok{*}\NormalTok{x }\SpecialCharTok{+} \DecValTok{2}
\NormalTok{  \}}
\NormalTok{  hessian }\OtherTok{\textless{}{-}} \ControlFlowTok{function}\NormalTok{(x) \{}
    \DecValTok{3}\SpecialCharTok{*}\NormalTok{x}\SpecialCharTok{\^{}}\DecValTok{2} \SpecialCharTok{{-}} \DecValTok{2}
\NormalTok{  \}}
  
  \ControlFlowTok{for}\NormalTok{(i }\ControlFlowTok{in} \DecValTok{1}\SpecialCharTok{:}\NormalTok{max\_iter) \{}
\NormalTok{    beta\_new }\OtherTok{\textless{}{-}}\NormalTok{ beta }\SpecialCharTok{{-}}\NormalTok{ (}\FunctionTok{grad}\NormalTok{(beta) }\SpecialCharTok{/} \FunctionTok{hessian}\NormalTok{(beta))}
\NormalTok{    betas\_vec[i}\SpecialCharTok{+}\DecValTok{1}\NormalTok{] }\OtherTok{\textless{}{-}}\NormalTok{ beta\_new}
    \ControlFlowTok{if}\NormalTok{(}\FunctionTok{abs}\NormalTok{(beta\_new }\SpecialCharTok{{-}}\NormalTok{ beta) }\SpecialCharTok{\textless{}=}\NormalTok{ stopping\_tol) \{ }\ControlFlowTok{break}\NormalTok{ \}}
\NormalTok{    beta }\OtherTok{\textless{}{-}}\NormalTok{ beta\_new}
\NormalTok{  \}}
\NormalTok{  iterations\_ran }\OtherTok{\textless{}{-}}\NormalTok{ i}
  \FunctionTok{return}\NormalTok{(}\FunctionTok{list}\NormalTok{(}\AttributeTok{iterations=}\NormalTok{iterations\_ran, }\AttributeTok{betas=}\NormalTok{betas\_vec))}
\NormalTok{\}}

\CommentTok{\# running the alg}
\ControlFlowTok{for}\NormalTok{ (starting\_point }\ControlFlowTok{in}\NormalTok{ starting\_points) \{}
\NormalTok{  result }\OtherTok{\textless{}{-}} \FunctionTok{newton\_pure\_alg}\NormalTok{(max\_iter, starting\_point, stopping\_tol)}
  \FunctionTok{cat}\NormalTok{(}\StringTok{"Starting Point:"}\NormalTok{, starting\_point, }\StringTok{"}\SpecialCharTok{\textbackslash{}n}\StringTok{Iterations:"}\NormalTok{, result}\SpecialCharTok{$}\NormalTok{iterations, }\StringTok{"}\SpecialCharTok{\textbackslash{}n}\StringTok{Betas:"}\NormalTok{, result}\SpecialCharTok{$}\NormalTok{betas,}\StringTok{"}\SpecialCharTok{\textbackslash{}n}\StringTok{"}\NormalTok{, }\StringTok{"\textasciitilde{}\textasciitilde{}\textasciitilde{}\textasciitilde{}\textasciitilde{}\textasciitilde{}\textasciitilde{}\textasciitilde{}\textasciitilde{}\textasciitilde{}\textasciitilde{}\textasciitilde{}\textasciitilde{}\textasciitilde{}\textasciitilde{}\textasciitilde{}\textasciitilde{}\textasciitilde{}\textasciitilde{}\textasciitilde{}\textasciitilde{}\textasciitilde{}\textasciitilde{}\textasciitilde{}\textasciitilde{}\textasciitilde{}\textasciitilde{}\textasciitilde{}\textasciitilde{}\textasciitilde{}\textasciitilde{}\textasciitilde{}\textasciitilde{}\textasciitilde{}\textasciitilde{}\textasciitilde{}\textasciitilde{}\textasciitilde{}\textasciitilde{}\textasciitilde{}\textasciitilde{}\textasciitilde{}\textasciitilde{}\textasciitilde{}\textasciitilde{}\textasciitilde{}\textasciitilde{}\textasciitilde{}\textasciitilde{}\textasciitilde{}\textasciitilde{}\textasciitilde{}\textasciitilde{}\textasciitilde{}\textasciitilde{}"}\NormalTok{, }\StringTok{"}\SpecialCharTok{\textbackslash{}n}\StringTok{"}\NormalTok{)}
\NormalTok{\}}
\end{Highlighting}
\end{Shaded}

\begin{verbatim}
Starting Point: -1 
Iterations: 8 
Betas: -1 -4 -2.826087 -2.146719 -1.842326 -1.772848 -1.769301 -1.769292 -1.769292 
 ~~~~~~~~~~~~~~~~~~~~~~~~~~~~~~~~~~~~~~~~~~~~~~~~~~~~~~~ 
Starting Point: 0 
Iterations: 20 
Betas: 0 1 0 1 0 1 0 1 0 1 0 1 0 1 0 1 0 1 0 1 0 
 ~~~~~~~~~~~~~~~~~~~~~~~~~~~~~~~~~~~~~~~~~~~~~~~~~~~~~~~ 
Starting Point: 0.1 
Iterations: 20 
Betas: 0.1 1.014213 0.07965577 1.009099 0.05222653 1.003965 0.02332944 1.000804 0.004806795 1.000035 0.0002072525 1 3.865288e-07 1 1.34559e-12 1 0 1 0 1 0 
 ~~~~~~~~~~~~~~~~~~~~~~~~~~~~~~~~~~~~~~~~~~~~~~~~~~~~~~~ 
Starting Point: 1 
Iterations: 20 
Betas: 1 0 1 0 1 0 1 0 1 0 1 0 1 0 1 0 1 0 1 0 1 
 ~~~~~~~~~~~~~~~~~~~~~~~~~~~~~~~~~~~~~~~~~~~~~~~~~~~~~~~ 
Starting Point: 2 
Iterations: 9 
Betas: 2 1.4 0.8989691 -1.288779 -2.105767 -1.8292 -1.771716 -1.769297 -1.769292 -1.769292 
 ~~~~~~~~~~~~~~~~~~~~~~~~~~~~~~~~~~~~~~~~~~~~~~~~~~~~~~~ 
\end{verbatim}

\subsubsection{Part (i) What do you
observe?}\label{part-i-what-do-you-observe}

\subsubsection{Part (ii) How can you fix the issue reported in
(i)?}\label{part-ii-how-can-you-fix-the-issue-reported-in-i}

\newpage

\subsection{Problem 2}\label{problem-2}

Consider the data in the train data.csv file. The first 600 columns
correspond to the predictors and the last column to the response y.

\subsubsection{Part (i) Implement that proximal gradient algorithm for
Lasso regression, by modifying appropriately your code from Homework
1.}\label{part-i-implement-that-proximal-gradient-algorithm-for-lasso-regression-by-modifying-appropriately-your-code-from-homework-1.}

To select a good value for the regularization parameter \(λ\) use the
data in the validation data.csv to calculate the sum-of-squares error
validation loss.

\subsubsection{\texorpdfstring{Part(ii) Show a plot of the training and
validation loss as a function of iterations. Report the number of
regression coefficients estimated as zero based on the best value of
\(λ\) you
selected.}{Part(ii) Show a plot of the training and validation loss as a function of iterations. Report the number of regression coefficients estimated as zero based on the best value of λ you selected.}}\label{partii-show-a-plot-of-the-training-and-validation-loss-as-a-function-of-iterations.-report-the-number-of-regression-coefficients-estimated-as-zero-based-on-the-best-value-of-ux3bb-you-selected.}




\end{document}
